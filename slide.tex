\documentclass{beamer}
\usepackage{hyperref}
\usepackage[T1]{fontenc}
\usepackage{latexsym,amsmath,xcolor,multicol,booktabs,calligra}
\usepackage{graphicx,pstricks,listings,stackengine}

\author{George Sittas}
\title{Monads in Functional Programming}
\institute{
    Department of Informatics and Telecommunications (DiT) \\
    University of Athens
}
\date{May, 2022}
\usepackage{Ritsumeikan}

\def\cmd#1{\texttt{\color{red}\footnotesize $\backslash$#1}}
\def\env#1{\texttt{\color{blue}\footnotesize #1}}
\definecolor{deepblue}{rgb}{0,0,0.5}
\definecolor{deepred}{rgb}{0.6,0,0}
\definecolor{deepgreen}{rgb}{0,0.5,0}
\definecolor{halfgray}{gray}{0.55}

\lstset{
    basicstyle=\ttfamily\small,
    keywordstyle=\bfseries\color{deepblue},
    emphstyle=\ttfamily\color{deepred},
    stringstyle=\color{deepgreen},
    numbers=left,
    numberstyle=\small\color{halfgray},
    rulesepcolor=\color{red!20!green!20!blue!20},
    frame=shadowbox,
}

\let\oldquote\quote
\let\endoldquote\endquote
\renewenvironment{quote}[2][]
  {\if\relax\detokenize{#1}\relax
     \def\quoteauthor{#2}%
   \else
     \def\quoteauthor{#2~---~#1}%
   \fi
   \oldquote}
  {\par\nobreak\smallskip\hfill(\quoteauthor)%
   \endoldquote\addvspace{\bigskipamount}}



\begin{document}

\begin{frame}
    \titlepage
    \begin{figure}[htpb]
        \begin{center}
            \includegraphics[keepaspectratio, scale=0.28]{pic/ekpa.jpg}
        \end{center}
    \end{figure}
\end{frame}

\begin{frame}
    \tableofcontents[sectionstyle=show,subsectionstyle=show/shaded/hide,subsubsectionstyle=show/shaded/hide]
\end{frame}

\section{Introduction}

\begin{frame}{What is a monad?}
    \begin{itemize}
        \setlength \itemsep{1em}
        \item An abstract mathematical concept, originally introduced in
              Category Theory by Roger Godement (1958).
        \item Use in Functional Programming due to Wadler, Moggi (1990).
        \item Software design pattern, which enables the
              \textbf{composition} of functions that take
              \textit{normal} values and wrap their results in a type
              (\textit{fancy value}) with additional \textbf{computation}
              (\textit{context}).
    \end{itemize}
\end{frame}

\begin{frame}{How does it help us?}
    \begin{itemize}[<+-| alert@+>]
        \setlength \itemsep{1em}
        \item Reduces code complexity by transforming complicated function
              sequences into succinct pipelines that abstract away control
              flow and side-effects (separation of concerns).
        \item Bridges pure and impure code (eg. I/O, mutable state) in pure
              functional languages, by isolating the side-effects.
        \item Makes semantics (\textit{effects}) explicit for a kind of
              computation, thus improving code clarity.
        \item Provides a toolkit for expressing and solving a wide range
              of problems, due to its abstract nature.
    \end{itemize}
\end{frame}

\section{Evaluating Expressions}

\begin{frame}[fragile]{Example: computations that can fail}
    Let's suppose that we want to write an expression evaluator. For
    simplicity, our expressions will only consist of integers that can
    be combined through division. In Haskell, we could express this as:

    \begin{minipage}{\linewidth} \hspace{1cm}
    \begin{lstlisting}[language=haskell, numbers=none, frame=none]
      data Expr = Val Int | Div Expr Expr
    \end{lstlisting}
    \end{minipage} \hspace{1cm} 

    In this way, we could represent the expression $5 \div 4 \div 3$ as:

    \begin{minipage}{\linewidth} \hspace{1cm}
    \begin{lstlisting}[language=haskell, numbers=none, frame=none]
      Div (Div (Val 5) (Val 4)) (Val 3)
    \end{lstlisting}
    \end{minipage} \hspace{1cm}

    Note: division is left-associative!
\end{frame}

\begin{frame}[fragile]{Example: computations that can fail}
    Given the above definition, we could then write a simple evaluator:

    \begin{minipage}{\linewidth} \hspace{1cm}
    \begin{lstlisting}[language=haskell, numbers=none, frame=none]
      eval :: Expr -> Int
      eval (Val n)   = n
      eval (Div x y) = (eval x) / (eval y)
    \end{lstlisting}
    \end{minipage} \hspace{1cm}

    Of course, this definition is bound to fail, were we to try and
    divide by zero. So, we would like to write a division-safe version
    instead:

    \begin{minipage}{\linewidth} \hspace{1cm}
    \begin{lstlisting}[language=haskell, numbers=none, frame=none]
      safeDiv :: Int -> Int -> Maybe Int
      safeDiv n m = if m == 0 then Nothing
                              else Just (n / m)
    \end{lstlisting}
    \end{minipage} \hspace{1cm}
\end{frame}

\begin{frame}[fragile]{Example: computations that can fail}
    \begin{minipage}{\linewidth} \hspace{1cm}
    \begin{lstlisting}[language=haskell, numbers=none, frame=none]
      eval :: Expr -> Maybe Int
      eval (Val n) = Just n
      eval (Div x y) = case eval x of
        Nothing -> Nothing
        Just n  -> case eval y of
            Nothing -> Nothing
            Just m  -> n `safeDiv` m
    \end{lstlisting}
    \end{minipage} \hspace{1cm}

    \begin{itemize}
        \item Program is now less readable: filled with boilerplate code.
        \item Pattern: when trying to chain together \textit{computations
        that might fail}, we first inspect the value we get from the first
        computation, and if it's a Just value, we feed it forward to\\
        the next computation, otherwise the whole computation fails.
    \end{itemize}
\end{frame}

\begin{frame}[fragile]{Example: computations that can fail}
    We can now abstract this pattern away, by defining a new operator
    which we will call "bind", that handles all the dirty work for us:

    \begin{minipage}{\linewidth} \hspace{1cm}
    \begin{lstlisting}[language=haskell, numbers=none, frame=none]
      bind :: Maybe Int -> (Int -> Maybe Int)
                        -> Maybe Int
      bind Nothing  f = Nothing
      bind (Just x) f = Just (f x)
    \end{lstlisting}
    \end{minipage} \hspace{1cm}

    Its job is to \textit{un-wrap} a Maybe value, and if it's a Just value,
    then apply whatever function it's passed as argument to it
    (here, it's f), and finally re-wrap the value inside a Maybe, in order
    to \textit{preserve the context of the computation} (possible failure).
\end{frame}

\begin{frame}[fragile]{Example: computations that can fail}
    The above code can then be transformed equivalently into:

    \begin{minipage}{\linewidth} \hspace{1cm}
    \begin{lstlisting}[language=haskell, numbers=none, frame=none]
      eval :: Expr -> Maybe Int
      eval (Val n) = Just n
      eval (Div x y) = eval x `bind` (\n ->
                       eval y `bind` (\m ->
                       n `safeDiv` m))
    \end{lstlisting}
    \end{minipage} \hspace{1cm}

    \begin{itemize}
        \item We reduced a portion of the boilerplate code, but in terms of
              readability, this version isn't \textit{a lot} better (yet).
        \item Here, the "bind" operator \textit{binds} x to n, and y
              to m, if both eval computations succeed, and then safeDiv
              is called.
        \item Inspecting the type signatures can help to understand this!
    \end{itemize}
\end{frame}

\begin{frame}[fragile]{Example: computations that can fail}
    We actually don't need to define the "bind" operator, because Haskell
    has already done it for us! Its type signature is (roughly):

    \begin{minipage}{\linewidth} \hspace{1cm}
    \begin{lstlisting}[language=haskell, numbers=none, frame=none]
      (>>=) :: m a -> (a -> m b) -> m b
    \end{lstlisting}
    \end{minipage} \hspace{1cm}

    \begin{itemize}
        \item m is a type constructor
        \item a is its type argument
        \item Describes the application of a function to a value that's
              contained in a context m, producing a value in the same
              context (of possibly different type)
    \end{itemize}

    \bigbreak
    So we can rewrite the above program as:
\end{frame}

\begin{frame}[fragile]{Example: computations that can fail}
    \begin{minipage}{\linewidth} \hspace{1cm}
    \begin{lstlisting}[language=haskell, numbers=none, frame=none]
      eval :: Expr -> Maybe Int
      eval (Val n) = Just n
      eval (Div x y) = eval x >>= (\n ->
                       eval y >>= (\m ->
                       n `safeDiv` m))
    \end{lstlisting}
    \end{minipage} \hspace{1cm}

    The nice thing is that Haskell also provides us with a shorthand
    notation, called "do notation", for code segments like this one.
\end{frame}

\begin{frame}[fragile]{Example: computations that can fail}
    This leads us to the final form of our initial program:

    \begin{minipage}{\linewidth} \hspace{1cm}
    \begin{lstlisting}[language=haskell, numbers=none, frame=none]
      safeDiv :: Int -> Int -> Maybe Int
      safeDiv n m = if m == 0 then Nothing
                              else Just (n / m)

      eval :: Expr -> Maybe Int
      eval (Val n) = Just n
      eval (Div x y) = do n <- eval x
                          m <- eval y
                          n `safeDiv` m
    \end{lstlisting}
    \end{minipage} \hspace{1cm}
\end{frame}

\begin{frame}[fragile]{Example: computations that can fail}
    \begin{itemize}
        \item Similar level of complexity to the initial, unsafe program.
        \item The failure management is now handled for us automatically.
    \end{itemize}

    \bigbreak
    We have essentially rediscovered the \textit{Maybe monad}!
    Right away, we can see how doing all this work helped us:

    \bigbreak
    \begin{itemize}
        \setlength \itemsep{1em}
        \item Reduce the complexity of the program, by abstracting away
              all additional computation in the definition of bind.
        \item Express the \textit{effect} of the computation in its type
              explicitly, namely: the evaluation of an expression
              \textit{might fail}.
    \end{itemize}
\end{frame}

\section{Defining the Monad}

\begin{frame}[fragile]{Overview}
    \begin{quote}{C. A. McCann}
        For a monad m, a value of type m a represents having access to a value of type a within the context of a monad.
    \end{quote}

    \begin{itemize}
        \item In the last example, we briefly mentioned the Maybe monad.
        \item A question now arises: \textit{what structure does a monad
        have}?
    \end{itemize}

    \bigbreak
    Let's see how we can answer this question in Haskell.
\end{frame}

\begin{frame}[fragile]{Definition}
    A monad can be created by defining a \textbf{type constructor} m of
    one type parameter a, and two operations:

    \bigbreak
    \begin{itemize}
        \item \textbf{return} (or \textit{unit}), which receives a value
        of type a and \textit{wraps} it into a \textbf{monadic value} of
        type m, using the type constructor:

        \begin{minipage}{\linewidth} \hspace{1cm}
        \begin{lstlisting}[language=haskell, numbers=none, frame=none]
          return :: a -> m a
        \end{lstlisting}
        \end{minipage} \hspace{1cm}

        \item \textbf{bind} (>>=), which receives a function f over a
        type a and can transform monadic values m a applying f to the
        unwrapped value a, returning a monadic value m b:

        \begin{minipage}{\linewidth} \hspace{1cm}
        \begin{lstlisting}[language=haskell, numbers=none, frame=none]
  (>>=) :: (m a) -> (a -> m b) -> (m b)
        \end{lstlisting}
        \end{minipage} \hspace{1cm}
    \end{itemize}
\end{frame}

\begin{frame}[fragile]{Definition}
    \begin{itemize}
        \setlength \itemsep{1em}
        \item The bind operator is a way to combine monadic values, and
              its precise definition differs, depending on the monad at
              hand.
        \item The return operator simply takes a value and puts it in some
        sort of default context -- a minimal context that still yields that
        value.
    \end{itemize}

    \bigbreak
    Note: return has no relation to the return \textit{statement} in
    imperative languages whatsoever. It's just a way to "promote" a
    \textit{normal} value into a \textit{fancy} (monadic) one.
\end{frame}

\begin{frame}[fragile]{Definition}
    The following \textbf{monadic laws} must also be satisfied:

    \bigbreak
    \begin{itemize}
        \item Left-identity:

        \begin{minipage}{\linewidth} \hspace{1cm}
        \begin{lstlisting}[language=haskell, numbers=none, frame=none]
          return a >>= h = h a
        \end{lstlisting}
        \end{minipage}

        \item Right-identity:

        \begin{minipage}{\linewidth} \hspace{1cm}
        \begin{lstlisting}[language=haskell, numbers=none, frame=none]
            m >>= return = m
        \end{lstlisting}
        \end{minipage}

        \item Associativity:

        \begin{minipage}{\linewidth} \hspace{1cm}
        \begin{lstlisting}[language=haskell, numbers=none, frame=none]
            (m >>= g) >>= h =
        = m >>= (\x -> g x >>= h)
        \end{lstlisting}
        \end{minipage} \hspace{1cm}
    \end{itemize}
\end{frame}

\begin{frame}[fragile]{Maybe monad, revisited}
    Having discussed about the general structure of a monad, we can
    finally see the complete definition of the Maybe monad in Haskell:

    \begin{minipage}{\linewidth} \hspace{1cm}
    \begin{lstlisting}[language=haskell, numbers=none, frame=none]
          instance Monad Maybe where
            return x = Just x
            Nothing >>= f = Nothing
            Just x >>= f = f x
    \end{lstlisting}
    \end{minipage} \hspace{1cm}
\end{frame}

\section{More Examples}

\begin{frame}[fragile]{The List monad}
    Haskell defines various monads, each representing a kind of
    computation with additional context or effects.

    \bigbreak
    The list type constructor is another classic example. Observe:

    \begin{minipage}{\linewidth} \hspace{1cm}
    \begin{lstlisting}[language=haskell, numbers=none, frame=none]
          instance Monad [] where
            return x = [x]
            xs >>= f = concat (map f xs)
    \end{lstlisting}
    \end{minipage} \hspace{1cm}
\end{frame}

\begin{frame}[fragile]{The List monad}
    But \dots \textit{how is a list a value with additional context}?
    One way to look at this, is that a list is a value that contains
    many values at the same time, something like \textit{a non-deterministic
    computation}.

    \bigbreak
    \begin{itemize}
        \setlength \itemsep{1em}
        \item return just takes a normal value and wraps it around a list,
              providing a minimal context.
        \item bind is more interesting, see the following example.
    \end{itemize}
\end{frame}

\begin{frame}[fragile]{The List monad}
    \begin{minipage}{\linewidth} \hspace{1cm}
    \begin{lstlisting}[language=haskell, numbers=none, frame=none]
        ghci> [3,4,5] >>= \x -> [x,-x]
        [3,-3,4,-4,5,-5]
    \end{lstlisting}
    \end{minipage} \hspace{1cm}

    Note: remember, the bind operator takes a monadic value and a function,
    and returns a monadic value in the same context (list)!

    \bigbreak
    \begin{itemize}
        \item bind takes a non-deterministic computation and feeds
              every result to the provided function, which results
              in a new non- deterministic computation
    \end{itemize}
\end{frame}

\begin{frame}[fragile]{The List monad}
    List comprehensions are actually syntactic sugar for using lists as
    monads! Observe:

    \begin{minipage}{\linewidth} \hspace{1cm}
    \begin{lstlisting}[language=haskell, numbers=none, frame=none]
[1,2] >>= \n -> ['a','b'] >>= \c -> return (n,c)

do { n <- [1,2]; c <- ['a','b']; return (n,c) }

[ (n,c) | n <- [1,2], c <- ['a','b'] ]
    \end{lstlisting}
    \end{minipage} \hspace{1cm}

    These three are equivalent. They all output the following:

    \begin{minipage}{\linewidth} \hspace{1cm}
    \begin{lstlisting}[language=haskell, numbers=none, frame=none]
        [(1,'a'),(1,'b'),(2,'a'),(2,'b')]
    \end{lstlisting}
    \end{minipage} \hspace{1cm}
\end{frame}

\begin{frame}[fragile]{The State monad}
    \begin{itemize}
        \item Another cool monad is the \textit{State} monad.
        \item It represents computations that need some kind of state
              (context) (examples: assignment, RNG, parsers, games).
        \item These computations can be thought of as functions that
              take a state and output a value and a new state with it:

        \begin{minipage}{\linewidth} \hspace{1cm}
        \begin{lstlisting}[language=haskell, numbers=none, frame=none]
               s -> (a,s)
        \end{lstlisting}
        \end{minipage} \hspace{1cm}

             Here, s is the state's type and a is the result's type.
        \item The point is to \textit{lazily} construct chains of
              functions that simulate changing state, yet in an
              immutable, referentially transparent way.
    \end{itemize}
\end{frame}

\begin{frame}[fragile]{The State monad}
    The State monad can be defined as follows:

    \begin{minipage}{\linewidth} \hspace{1cm}
    \begin{lstlisting}[language=haskell, numbers=none, frame=none]
newtype State s a = State { runState :: s -> (a,s) }

instance Monad (State s) where
  return x = State (\s -> (x,s))
  (State h) >>= f =
    State (\s -> let (a,s')    = h s
                     (State g) = f a
                 in g s')
    \end{lstlisting}
    \end{minipage} \hspace{1cm}

    \begin{itemize}
        \item return: take a value and make a \textit{stateful computation}
              that always has that value as its result.
        \item bind: get the current state, feed it to a computation f and
              get a new \textit{stateful computation} as output, with its
              state updated accordingly.
    \end{itemize}
\end{frame}

\begin{frame}[fragile]{The State monad}
    Here, runState is a way of executing such a computation: we pass a
    state to it and it executes each consecutive stateful computation,
    threading each new state to the next function each time. It returns
    the \textit{actual value} of the computation and the resulting state.

    \bigbreak
    There are two useful functions operating on stateful computations:

    \begin{minipage}{\linewidth} \hspace{1cm}
    \begin{lstlisting}[language=haskell, numbers=none, frame=none]
      get = State ( \s -> (s,s) )
      put s = State ( \s -> ((),s) )
    \end{lstlisting}
    \end{minipage} \hspace{1cm}

    \begin{itemize}
        \item get: take current state and present it as the result
        \item put: take some state and make a stateful function that
              replaces the current state with it.
    \end{itemize}
\end{frame}

\begin{frame}[fragile]{The State monad}
    Let's see a simple string parsing game as an example:

    \bigbreak
    \begin{itemize}
        \item Produce a number from a string of a's, b's and c's.
        \item By default, the game is off, "c" toggles the game on and off.
        \item An "a" gives +1, and a "b" gives -1.
    \end{itemize}
\end{frame}

\begin{frame}[fragile]{The State monad}
    \begin{minipage}{\linewidth} \hspace{1cm}
    \begin{lstlisting}[language=haskell, numbers=none, frame=none]
type GameValue = Int
type GameState = (Bool, Int)

playGame :: String -> State GameState GameValue
playGame [] = do
    (_, score) <- get
    return score

playGame (x:xs) = do
    (on, score) <- get
    case x of
      'a' | on -> put (on, score + 1)
      'b' | on -> put (on, score - 1)
      'c' |    -> put (not on, score)
      _        -> put (on, score)
    playGame xs
    \end{lstlisting}
    \end{minipage} \hspace{1cm}
\end{frame}

\begin{frame}[fragile]{The State monad}
    And we can then play the game like so:

    \bigbreak
    \begin{minipage}{\linewidth} \hspace{1cm}
    \begin{lstlisting}[language=haskell, numbers=none, frame=none]
ghci> fst (runState (playGame "cabca") (False, 0))
0
    \end{lstlisting}
    \end{minipage} \hspace{1cm}

    \bigbreak
    Which is the expected result. If it's not easy to see what's going
    on, maybe applying currying and lazy evaluation will help to grasp
    the execution flow.
\end{frame}

\begin{frame}[fragile]{Recap}
    Again, we can see the benefits that monads give us, as we discussed
    earlier:

    \bigbreak
    \begin{itemize}
        \setlength \itemsep{1em}
        \item In lists, abstracting away the chaining of concat and map
              gave us a pretty elegant tool, which enhances the program's
              expressibility: list comprehensions
        \item In stateful computations, we have abstracted away the
              threading of the states, creating a sense of "mutable"
              state in our program, even though it's all pure under
              the hood!
        \item The types express the computational effects clearly: with
              lists, each computation can have 0 or more values, whilst
              with stateful computations, additional state is needed.
    \end{itemize}
\end{frame}

\section{Closing Remarks}

\begin{frame}[fragile]{Monads in the wild}
    Monads can be found in various different pieces of software. For
    instance, here are a couple of them:

    \bigbreak
    \begin{itemize}
        \setlength \itemsep{1em}
        \item Parsing: the Parsec library uses monads to combine simpler
              parsing rules into more complex ones.
        \item LINQ by Microsoft provides a query language for the .NET
              framework that is heavily influenced by functional prog.
              concepts, including operators for composing queries
              monadically.
    \end{itemize}
\end{frame}

\begin{frame}[fragile]{Another point of view}
    Typically, programmers will use the bind operator to chain
    \textit{monadic functions} into a sequence, which has led
    some to describe monads as \textbf{programmable semicolons},
    which references the way other languages (imperative) use
    semicolons to separate statements.
\end{frame}

\begin{frame}[fragile]{References}
    \begin{itemize}
        \item Learn You A Haskell For Greater Good
        \item Wikipedia - Monads
        \item Haskell wiki - All About Monads
        \item Computerphile - What is a Monad?
    \end{itemize}
\end{frame}

\end{document}